\section{Introduction}
\label{intro}
%The ultimate goal of robotics is to enable robots to be a part of human life, to provide assistance in a wide range of tasks including surgery operation, house keeping, education and personal care. Most of these tasks require the ability to do complex object manipulation, which remains a open challenge in the robotics community. %It is impractical to pre-program all such object manipulation tasks manually. Therefore, we use a more practical alternative: teaching the robots by human demonstrations of the tasks.

With robots moving into human-centered environments such as households
and offices, human-like motor skills are becoming increasingly
desirable. In everyday life, object manipulation is one of the most
commonly used manual skills. Object manipulation includes a large
category of activities ranging from the simple pick-and-place task to
complicated dexterous manipulation.  

Here we provide a framework for learning a human object-manipulation skill
and transfering it to a robot.
%Object manipulation is a large category of tasks in which the robots make physical contact with the environment. %Typical contact tasks include robot grasping and object manipulation.
%% The complicated physics in the interactions between objects makes manipulation tasks difficult. The multi-body interaction and the effect of friction can cause abrupt changes in the environment. To handle these situations, robots have to be equipped with a nonlinear and non-stationary control strategy, which is hard to design by analytical approaches.
%In manipulation tasks, the ability to predict and react to the uncertainty and unforeseen changes of the environment is vital.
Generally, manipulation tasks are very difficult, due to the
complicated contact situations between the manipulator and the object,
and the changing kinematic and dynamic contexts that result. Humans
can perform these skilled tasks and adapt to changes in context
without difficulty. At the heart of this skill is
prediction~\citep{flanagan2006control}. Studies from neuroscience
suggest that humans develop internal models for motor control, which
allow us to predict the future state of the environment. By comparing
the predictive state with the actual sensory state, the internal
models monitor the progression of  tasks, and launch any
corresponding motor correction and motor reaction required to
adapt to anything unexpected. %Building accurate analytical model for manipulation task is painstaking, small error in the model will cause fatal control problem. To solve this problem, we take the programming by human demonstration (PbD) approach.  %Modular approach has also been shown to be an effective way of building intelligent systems~\citep{bryson2004modular,BrysonMcG12}.

Inspired by this concept, we propose an approach to learning human
adaptive control strategies. This strategy is encoded with a modular
model, where each module includes a forward model for context
estimation, and an inverse model for motor command generation. From
multiple human demonstrations, we extract a set of strategies, each of
which takes charge of one specific task context. % JJB: Bidan, a lot of
                                % this is redundant.  Be careful to
                                % only say something once at least in
                                % the same paragarph!  By this method, we
%modularize human adaptive control strategy. 
The internal forward and inverse models are learnt within each
module with a representation that can be easily transfered to a
robot. When the robot executes a similar task, the forward models
estimate the context of the task and 
%`contextized' ??? JJB
`contectualize' the inverse models, allowing them to generate the proper commands.
% JJB??? to drives the object.
%Each learnt control strategy is weighted by the similarity of the current task context and its corresponding context. The optimal strategy is computed as the linear combination of the weighted commands of each strategy.

%it carries out an automatic estimation of the task context based on the sensory data and calculates a best combination of the modules to provide a motor command of the current operating region.
%The approach does not require any prior knowledge of the kinematics nor dynamics of the operation system, nor is it restricted to a specific robot platform. The control strategy is learnt on the object level and hence can be transfer from human to robot directly.

Our work contributes a framework composed of both automated and
bespoke components for creating the modular representation of 
human adaptive control-strategies %JJB redundant! of manipulation
                                %tasks 
and to transfer these learnt internal models to a robot. To verify our
approach, we use an \emph{Opening Bottle Caps} task as an
experiment. An adaptive control strategy is required here, because the
friction between the bottle's and the cap's surfaces has multiple
phases. We demonstrate the modularized version of the human control
strategy in this task on a robot, which is used to open both familiar
and novel bottles.

%This task is complex in the sense that 1) there are multiple different contacts occur at the same time, i.e., the contact between the fingertips and the cap, the contact between the cap and the bottle screw; 2) the latter contact usually consists of several abrupt switch during the task, from static friction to kinetic friction.
The rest of this article is organized as follows:
Section~\ref{sec:related} provides an overview of related work;
Section~\ref{sec:method} presents our approach of learning a
multiple-module model of a human manipulation strategy. The
experiments on the opening-bottle-cap task and their result are shown
in Section~\ref{sec:exp}, along with details of the hardware
specifications and the experimental setup.  Section~\ref{sec:diss}
concludes this work with a discussion and a look towards future work.
