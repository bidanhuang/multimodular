\section{Introduction}
\label{intro}
%The ultimate goal of robotics is to enable robots to be a part of human life, to provide assistance in a wide range of tasks including surgery operation, house keeping, education and personal care. Most of these tasks require the ability to do complex object manipulation, which remains a open challenge in the robotics community. %It is impractical to pre-program all such object manipulation tasks manually. Therefore, we use a more practical alternative: teaching the robots by human demonstrations of the tasks.
With robots moving into human-centered environments, such as household, working office, it becomes more and more desirable to endow robots with the human-like ability. In everyday life, object manipulation is one of the most commonly used skills, which includes a large category of activities ranging from the simple pick-and-place task to the complicated dexterous manipulation task.


%Object manipulation is a large category of tasks in which the robots make physical contact with the environment. %Typical contact tasks include robot grasping and object manipulation.
%% The complicated physics in the interactions between objects makes manipulation tasks difficult. The multi-body interaction and the effect of friction can cause abrupt changes in the environment. To handle these situations, robots have to be equipped with a nonlinear and non-stationary control strategy, which is hard to design by analytical approaches.
%In manipulation tasks, the ability to predict and react to the uncertainty and unforeseen changes of the environment is vital.
<<<<<<< HEAD
Generally, these manipulation tasks are very difficult due to the complicated contact situations and the changing kinematic and dynamic contexts. Despite this, humans can perform such skilled tasks and adapt to the changes without difficulty. At the heart of this skill is prediction~\cite{flanagan2006control}. Studies from neuroscience suggest that human develop internal models for motor control, to predict the next state of the environment. By comparing the predictive state with the actual sensory state, the internal models monitor the progression of the tasks and launch the corresponding motor correction and motor reaction. One hypothesis of this mechanism is MOSAIC~\cite{haruno2001mosaic}, which is a modular model composed by multiple pairs of forward model and inverse model. %Modular approach has also been shown to be an effective way of building intelligent systems~\cite{bryson2004modular,BrysonMcG12}.


Inspired by this model, we propose a modular approach that encodes human manipulation skill and transfer to a robot. From multiple human demonstrations, we extract a set of control strategies. Each strategy take charge of one task context, composed by a forward model for context estimation and an inverse model for command generation. When a robot executes a similar task, the forward models estimate the context of the task based on the sensory data and ``contextized'' the inverse models to generate proper command that drives the object to the desire state.
%Each learnt control strategy is weighted by the similarity of the current task context and its corresponding context. The optimal strategy is computed as the linear combination of the weighted commands of each strategy.
=======
Generally, these manipulation tasks are very difficult due to the complicated contact situations and the intrinsic uncertainties, such as coefficient of friction. However, humans can perform such skilled tasks very easily without acquiring the exact knowledge of the environment. At the heart of this skill is prediction~\cite{flanagan2006control}. Studies from neuroscience suggest that human develop internal models for motor control, to predict the next state of the environment. By comparing the predictive state with the actual sensory state, the internal models monitor the progression of the tasks and launch the corresponding motor correction and motor reaction. One hypothesis of this mechanism is MOSAIC~\cite{haruno2001mosaic}, which is a modular model composed by multiple pairs of forward model and inverse model. %Modular approach has also been shown to be an effective way of building intelligent systems~\cite{bryson2004modular,BrysonMcG12}.


Inspired by this model, we propose a modular approach that encodes human manipulation skill and transfer to a robot.
Our approach firstly extracts a set of control strategies from human demonstrations in different task contexts. Each strategy take charge of a certain group of task contexts. These strategies encode the correlation between the exert force and object displacement, which can generate proper commands to drive the object to the desire state.
When a robot executes a similar task, based on the sensory data, it estimates the context of the task and searches for an optimal strategy from the learnt experience. Each learnt control strategy is weighted by the similarity of the current task context and their corresponding contexts. The optimal strategy is computed as the linear combination of the weighted commands of each strategy.
>>>>>>> 02b51e79a4c9f856083586a8b164b8e077f207df

%it carries out an automatic estimation of the task context based on the sensory data and calculates a best combination of the modules to provide a motor command of the current operating region.
%The approach does not require any prior knowledge of the kinematics nor dynamics of the operation system, nor is it restricted to a specific robot platform. The control strategy is learnt on the object level and hence can be transfer from human to robot directly.
To verify our approach, we use \emph{Opening Bottle Caps} as an experimental example. This task is complex in the sense that the friction between the bottle and the cap surfaces has multiple phases and the transition between them is hard to predict. Adaptive control strategy is required for this task. To this end, we use our modular approach to learn the strategy from human and implementing it on a robot to open learnt and unseen bottles.

<<<<<<< HEAD
%This task is complex in the sense that 1) there are multiple different contacts occur at the same time, i.e., the contact between the fingertips and the cap, the contact between the cap and the bottle screw; 2) the latter contact usually consists of several abrupt switch during the task, from static friction to kinetic friction.
=======
To verify our approach, we use \emph{Opening Bottle Caps} as an experimental example. This task is complex in the sense that the friction between the bottle and the cap surfaces has multiple phases, such as static friction and kinetic friction. The physics of these phases, as well as the mechanism of how do them switch is not fully understood. Despite this, human is able to open bottle caps using adaptive control strategy. We learn this strategy by our modular approach and implementing it on a robot to open learnt and unseen bottles.

%This task is complex in the sense that 1) there are multiple different contacts occur at the same time, i.e., the contact between the fingertips and the cap, the contact between the cap and the bottle screw; 2) the latter contact usually consists of several abrupt switch during the task, from static friction to kinetic friction.

>>>>>>> 02b51e79a4c9f856083586a8b164b8e077f207df
The rest of the paper is organized as follows: Section~\ref{sec:related} provides an overview of the related work; Section~\ref{sec:method} presents our approach of learning a multiple module model of human manipulation strategy; an experiment of a opening bottle cap task and its result is shown in Section~\ref{sec:exp}, along with details of the hardware specifications and the experimental setup; Section~\ref{sec:diss} concludes this work with an outlook of future work. 