\section{Introduction}
\label{intro}
%The ultimate goal of robotics is to enable robots to be a part of human life, to provide assistance in a wide range of tasks including surgery operation, house keeping, education and personal care. Most of these tasks require the ability to do complex object manipulation, which remains a open challenge in the robotics community. %It is impractical to pre-program all such object manipulation tasks manually. Therefore, we use a more practical alternative: teaching the robots by human demonstrations of the tasks.
With robots moving into human-centered environments, such as household, working office, it becomes more and more desirable to endow robots with the human-like ability. In everyday life, object manipulation is one of the most commonly used skills, which includes a large category of activities ranging from the simple pick-and-place task to the complicated dexterous manipulation task.


%Object manipulation is a large category of tasks in which the robots make physical contact with the environment. %Typical contact tasks include robot grasping and object manipulation.
The complicated physics in the interactions between objects makes manipulation tasks difficult. The multi-body interaction and the effect of friction can cause abrupt changes in the environment. To handle these situations, robots have to be equipped with a nonlinear and non-stationary control strategy, which is hard to design by analytical approaches.
%In manipulation tasks, the ability to predict and react to the uncertainty and unforeseen changes of the environment is vital.

The excellent manipulation skill of human is long desired by roboticists. Without using theoretical knowledge of the dynamics of the complex environment, human can manipulate objects easily.
At the heart of this skill is prediction~\cite{flanagan2006control}. Evidences of neuroscience suggest that human develop internal models for motor control, to predict the next state of the environment. By comparing the predictive state with the actual sensory state, the internal models monitor the progression of the tasks and speed up correction and reaction of the motor control. This inspires us to learn the internal models that human use for manipulation.

The fact that human have no difficulties in controlling their limbs under changes of the environment suggests that the brain use more than one models. One hypothesis of this mechanism is MOSAIC~\cite{haruno2001mosaic}, which is a multiple modular model composed by a couple of pairs of forward model and inverse model. We build our control strategy based on this hypothesis. Modular approach has been shown to be an effective way of building intelligent systems~\cite{bryson2004modular,BrysonMcG12}. In this work, we introduce a modular approach to learn human internal models for manipulation. Our approach firstly identifies a set of control strategies from human demonstration, each adapted to different task contexts. When a robot executes a task, it carries out an automatic estimation of the task context based on the sensory data and calculates a best combination of the modules to provide a motor command of the current operating region. The approach does not require any prior knowledge of the kinematics nor dynamics of the operation system, nor is it restricted to a specific robot platform. %The control strategy is learnt on the object level and hence can be transfer from human to robot directly.

We verify this approach experimentally by learning a specific task: opening bottle caps. This is a complex manipulation task as the friction between the contact surfaces has multiple phases of which the physics behind is not completely understood. With little knowledge of tribology, human is able to open many different bottle caps using adaptive control strategy. We learn this strategy by our modular approach. The robustness of the learnt control strategy is confirmed by applying it to open different bottles including unseen bottles. We show that our method simplifies the learning of a control strategy and can benefit a wide range of tasks.


The rest of the paper is organized as follows: Section~\ref{sec:related} provides an overview of the related work in literatures that motivate this work; Section~\ref{sec:method} introduces our approach of learning a multiple module model of human manipulation strategy; an experiment of a opening bottle cap task and its result is shown in Section~\ref{sec:exp}, along with details of the hardware specifications and the experimental setup; Section~\ref{sec:diss} concludes and discusses the contribution of this work, as well as proposes directions for future study. 