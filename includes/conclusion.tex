\section{Conclusion and Discussion}
\label{sec:diss}
In this paper we proposed a modular approach for learning manipulation task from human demonstration. We find out the number of modules needed in a task by hierarchical clustering. From each cluster we use forward and inverse model pairs to model the motor control mechanism. The forward models are to predict the effect of the previous motor command, while the inverse models compute a motor command to bring the current state to a desired state. The statical approach enables us to estimate the reliability of the inferences of each module under the current context. The final motor command is the sum of the weighted command from each module. Our experiments verify that by this modular approach, the robot can automatically recognize the current task context and compute motor commands to accomplish a manipulation task, i.e. opening bottle caps.

%We contribute an experimental validation of this approach by learning the human strategy of opening bottle cap. This is a complex manipulation task as the friction between the contact surfaces involve extremely complicated physics. Without detail knowledge of tribology, human success to open many different bottle caps in daily life. We aim to learn this strategy by a modular approach. The human demonstrator demonstrated the task in seven different contexts and we group these demonstrates into three groups. Forward and inverse model pairs are learnt from each group for motor control.

There are many promising directions of further studies of this work. The first is to apply this approach to more contact and friction relative tasks and learn a more general human control strategy to handle the instability caused by friction. To extend our approach to learn tasks involve multiple steps, one could also integrate it with task segmentation technique, to break down the task into atomic steps and recognize the steps needs modular approach. So far we have implement the approach with a task controlled in three dimensions and by three modules. How does the number of modules change according to the dimension of the task is another useful information to reason about.

To summary, tasks involve multiple phases or contexts are hard to implement by a single model. By clustering the control strategies and corresponding task contexts, we are able to focus on each task subspace and build local models. Modular architecture is a practical approach for modeling these tasks. As manipulation usually involves multi-phase friction and multi-body interaction, learning manipulation tasks with a modular approach can simplify the modeling problem in a large extend.
%\begin{itemize}
%  \item Can be used in more complex robot
%  \item if slip can be detected ...
%  \item grasp the cap, studied in another experiment
%\end{itemize}

%During the task, the dynamics of the environment changes abruptly. Figure~/ref{phase12} shows an example of the recorded time sequence of the opening bottle cap task. From the 4 turning cycles we see a dramatic difference between the first cycle and the rest of the cycles. This is not surprise as in the first turning cycle we have to apply a large torque to break the contact between the cap and the bottle, i.e. to overcome the static friction. Once the contact is broken, a much smaller torque is required to rotate the cap, i.e. to overcome the kinema.tic friction.
