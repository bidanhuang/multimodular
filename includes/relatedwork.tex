\section{Related Work}
\label{sec:related}
In this section, we give a overview of the area of the
machine-learning of manipulation tasks for robots, and of modular
approaches.


%\subsection{Robot learning}
%\label{sec:imitation}
%Demonstration based learning has been an active topic of research in the last decade. While manually programming can be tedious, especially for unstructured and uncertain environment, program by demonstration (PbD) provide a way to speed up the process. It is designed to automatically extract key features and task constraints, which are invariance to environmental changes. This approach has been extensively studied in literatures~\citep{calinon2007learning,dillmann2004teaching,kulic2012incremental}.
%The approach of PbD usually goes through three stages: perceive (demonstration), recognise (skill modeling) and reproduction (task implementation)~\citep{demiris2002f}. The advantages of PbD largely come from the demonstration and skill modeling steps. In the later part of this section, we provide an overview on these two stages.

\subsection{Learning Manipulation Tasks}
Demonstration-based learning has been extensively
studied~\citep{calinon2007learning,dillmann2004teaching,kulic2012incremental}
as a promising approach for building robot
intelligence. %It is essential for the tasks that analytical expression of the system is hard to derive.
Learning manipulation tasks is one of the main application of this
approach. The physical properties of a manipulation task is hard to
express analytically, and as a result the control strategy is hard to
derive formally without examples. Modeling an expert's demonstration
of strategies has been used as an alternative to fully analytical
solutions.

Two major forms of demonstration are used in teaching manipulation
tasks: kinematics teaching and tele-operation. In kinesthetic FIXME
kinematic or kinesthetic?  be consistent % FIXME
teaching, a human directly contacts the robot and guides the robot's
movements to accomplish a
task~\citep{korkinof2013online,pais2014encoding,pastor2011skill,Miao2014}. The
trajectory of movements and contact force are recorded by the robot's
sensors.
% ===== Why not kinematics approach? =====
This method is simple and effective, but it is limited in the number
of controllable end effectors that can be
trained. %FIXME is this right?
While a manipulation task usually involves multifinger movement, a
human can kinematically operate one finger with each hand and hence
two fingers simultaneously at most. To control multi-finger hands,
some researchers use
tele-operation~\citep{bernardino2013precision,kondo2008recognition,Fischer98}. This
usually relies on data gloves or other motion-capture systems, which
sense the human hand and arm motions. The human motion is mapped to
the robot's to generate motions in the robot in real time, allowing
the robot to record its own interactions with the
environment. %JBB I added a lot of clarification here
In fine manipulation tasks, the robot platforms are usually restricted
to anthropomorphic hands for better mapping. One drawback of
teleoperation methods is that they typically provide no direct force
feedback to the human demonstrator during manipulation.  This can make
it difficult for the human to provide dexterous manipulation examples
for recording.
% FIXME right?  this is the problem? (I added a sentence for clarity,
% if it's wrong, add a diffferent one.) Also, surely datagloves do
% allow you to get force feedback -- record people making natural
% motions on real objects?  So do many other motion capture systems --
% it's only teleoperation that is unlikely to provide haptic feedback?

Another approach involves the human demonstrating manipulation tasks
with their own bodies, rather than directing the
robot~\citep{asfour2008imitation}. With direct interaction with the
object the human demonstrator is able to perform the task most
naturally and with a more delicate control strategy. However, the task
information captured from these human demonstrations must then be
transferred to robots. Various methods for mapping between human and
robot articulation have been
proposed~\citep{hueser2006learning,asfour2008imitation,do2011towards,}. These
may be augmented with 
correction by humans~\citep{calinon2007incremental,sauser2011iterative,romano2011human}
and by self-correction via learning~\citep{bidan2013robio}.% are proposed
%as alternative solutions. JJB: these aren't really alternatives, they
%are complementary.
In general, the effective transfer of human skills to robots skills
remains a challenge.

%FIXME check this over, I changed a bunch. 
Our proposed method derives from this last class of demonstrations.
We allow the subject to perform a manipulation task directly on an
object and experience natural feedback.  Our contribution is an
ability to capture both the control and the expected feedback from the
demonstration, and to encode the strategy in a way that can then be
easily transferred to any robot platform. In our task demonstration, a
human wears tactile sensors for example on a glove, and directly
interacts with objects. The demonstration is recorded and expressed
from an object-centric viewpoint. The object-centric
viewpoint~\citep{okamura2000overview,jain2013improving,Miao2014}
centers the representation of the manipulation task on the manipulated
object (a form of allocentrism), rather than on the robot (egocentrism). This
suggests that the goal of a manipulation task is to produce a desired
object movement rather than a robot end-effector movement. Our
approach takes this principle and learns a control strategy for
producing a desired object behavior.  The demonstrated strategy
expressed from the object perspective can then be transferred to a
robot platform by converting %FIXME?  Was "covering "
the exerted force to robot joint torque.

With the object centric viewpoint, we need to learn the correlation
between the exerted force of the object and the desired object
motion. A classic model of this correlation is
impedance~\citep{howard2010transferring,wimbock2012comparison}. Given
the desired impedance of a task, we can compute proper motor commands
for the robot to accomplish it. Fix impedance control is limited to
simple tasks. In many manipulation tasks such as opening a bottle cap,
variable impedance is required: at the beginning we need a large
impedance to break the contact between the bottle and the cap, and
later we need a small impedance to drive the cap smoothly. For such
tasks, fixed impedance control will either lead to task failure or cause
hardware damage.  However, computing the impedance for a given task
involving variable impedance is difficult.  In many cases the
impedance is roughly approximated by a linear model, but this is
inadequate for nonlinear tasks. %FIXME citations for this would be
                                %nice, though not necessary.


Variable impedance can be learnt by a human physically correcting the
robot's impedance---for example, wiggling the robot's arm---in different stages of
the task~\citep{kronander2012online}. For learning manipulation,
however, wiggling the robot's fingers will interrupt the task and may
cause task failure.
%This is feasible for learning robot arm impedance but not for object impedance.
Variable impedance can also be learnt with the reinforcement learning
algorithm Policy Improvement with Path Integrals ($PI^2$) with a task
specific cost function~\citep{buchli2011learning}. Designing this cost
function requires insight into the task and usually is
difficult. Therefore, in our approach we directly model the
correlation of the exerted force and the object motion by a nonlinear
statistical model.


% GMM
%In our approach we directly model the correlation of the exerted force and the object motion using Gaussian Mixture Model (GMM)~\citep{cohn1996active}. GMM has been shown to be an efficient model in capturing the nonlinearity of data~\citep{huang2013learning,sauser2011iterative,calinon2007incremental}. Our choice of GMM has other merits which will be discussed in later sections.
% benefit of using GMM will be discussed in later sections.

% deleted "Among" -- that would indicate some, not all: JJB
In the approaches mentioned above, a single model is built for the
entire task. For tasks involving of multiple phases of dynamics, such
as are generated by friction, a single control strategy may be inadequate. To handel
varying task contexts, robots operating in human-centric environments
need to be equipped with multiple strategies. In the next section we
give a brief overview of  modular and multiple model approaches.


\subsection{Modular Approaches to Learning Manipulation}
% Some manipulation modelling method. drawback. You must survey other approaches to model multiple sensori-motor loops (also called motor program, motor primitives, behaviors, etc) and switching across these models, and then justify the use of Mosaic

%% Internal model
%In the study of human motor control, internal model is one of the most evidently supported hypothesis. It postulates a neuron process that simulates output of the motor system and the environment~\citep{kawato1999internal}. It's applications in robot control have been explored by many researchers. Various types of internal models are built for different tasks~\citep{sciavicco2000modelling,jordan1992forward}.

% Adaptive
Modular approaches are widely used in adaptive control and their
benefit has been long
discussed~\citep{jacobs1991adaptive,narendra1997adaptive,BrysonIJCAI01}.
Modular approaches have previously been shown to be an effective
architecture of building intelligent
systems~\citep{mobile98,BrysonJETAI00}.  In manipulation tasks,
context changing is a common phenomenon due to object
interactions. These changes are often rapid or discontinuous. Classic
adaptive control approaches such as model
identification~\citep{khalil2004modeling} are inadequate for these
tasks, as instability or error may occur during the optimization of
the model variables. To quickly adapt, the multiple model
approach~\citep{narendra1995adaptation}, referred to here as the
modular approach, is proposed. In this approach, multiple controllers
are designed, each of which is in charge of a certain task
context. During control, the task context is estimated online and the
corresponding controllers are activated. Some authors have relatively
recently presented promising module-based approaches for learning as
well~\citep{fekri2007robust,kuipers2010multiple}.  In robotics,
modularity is particularly useful for tasks in non-stationary
environments~\citep{sugimoto2012emosaic}.

% Modular - human MOSAIC

%%The excellent ability of human to manipulate different objects in different contexts and to quickly adapt to change of contexts suggested that our central nervous system (CNS) maintains multiple internal models            of outside environments, rather than a single internal model that adapts to new environment~\citep{neilson1985acquisition}.
%Based on the concept of modularity, the Modular selection and identification for control (MOSAIC) architecture was proposed to explain the motor control strategy of the CNS~\citep{wolpert1998multiple}. According to this hypothesis, during the motor control process, the brain quickly select the most appropriate modules according to the current environment context and use them to generate an appropriate motor command to react to the environment~\citep{haruno2001mosaic}.

% Our approach
The modular approach we describe here is inspired by
MOSAIC~\citep{haruno2001mosaic}. MOSAIC (MOdular Selection And
Identification for Control) is a paradigm of multiple-module control,
where each module is composed of a forward model and an inverse
model. The forward models are responsible for estimating the task
context in real time, and the inverse models are used to generate
appropriate motor commands for the current context. The inverse models
are weighted by the accuracy of the estimations of their corresponding
forward models. The final motor command is the linear combination of
the commands factored by their weights.

We take the paradigm of MOSAIC but implement the modular model in our
own manner. In earlier work, \citet{wolpert1998multiple} used
Artificial Neural Network (ANN) to encode the internal models,
i.e. the forward models and the inverse models. The variance of a
forward model, which decides how much the multiple modules
collaborate, has to be manually tuned. MOSAIC addresses this
hand-tuning problem by modeling the transition between modules using a
Hidden Markov Model~(HMM) and optimizing the variance with the
Expectation Maximization~(EM) algorithm~\citep{haruno2001mosaic}. In
this method the forward models are approximated by linear systems. In
order to
solve the hand tuning problem of the variance but without restricting the
complexity of the internal models, we encode our internal models with
Gaussian Mixture Models (GMM)~\citep{cohn1996active}.  Training the
GMM by the EM algorithm, we compute the optimal values of the models'
parameters. GMM has been shown to be an efficient model in capturing
the nonlinearity of
data~\citep{calinon2007incremental,sauser2011iterative,huang2013learning}.
This allows us to approximate more complex internal models.

%We take this paradigm and model our internal models by GMM. Training GMM with the Expectation Maximization algorithm (EM), we estimate the optimal values of the model parameters. Compare to the early work~\citep{wolpert1998multiple} which use Neural Network and has to manually tune the variance of each forward model, GMM has the advantage of automatically computing the all the model parameters. Later work~\citep{haruno2001mosaic} fixes the hand tuning problem by modeling with Hidden Markov Model (HMM) and optimize the model with EM. With this method the forward models are assumed to be linear. In our approach, GMM allows non-linear system to be modeled. Figure.~\ref{fig:control} illustrates the workflow of our approach. Compare to the switching modular method~\citep{narendra1997adaptive}, i.e. only one module will be activated and used to generate motor command, the linear combination of the modules requires less number of modules to approximate the system dynamics.
%The main reason is that neuron network can not directly estimate the likelihood of one query point. With NN, the the variance in each forward model, which is an essential variable to control how much the modules  cooperate, has to be hand tuned. .


In another approach, \citet{petkos2006learning} have united the
forward model and inverse model into
a single model. For a particular task % FIXME you said "that" not "a", but you
                                % haven't introduced a particular
                                % task.  If you want to talk about one
                                % particular task, you need to
                                % introduce it in a sentence first,
                                % not only cite the paper.  All we know
                                % about that paper right now is that it
                                % introduces a learning model, not
                                % that it describes any tasks.
the action~($a_t$) taking the current task state~($s_t$) to the
desired task state~($s_{t+1}$) is always unique. However, in many
cases this mapping is not unique and hence the inverse model has to
include extra variables in order to resolve the non-uniqueness. To
take a more general approach, we build the forward and inverse model
separately. %JJB if there are multiple solutions, does it matter which
            %one is chosen?  Why not just pick one?  Or is that
            %impossible because of the representation, you just get a
            %blend / chaos?
%However this model does not provide a method to find out number of models needed in tasks. %and the inverse model represented by the joint distribution will potentially produce an invalid command that average all possible solutions.

In our model, the final motor command is the sum of the outputs of all
modules. This stands in contrast to the switching modular
method~\citep{narendra1997adaptive}, where only one module will be
activated to generate motor commands per time step.  Our approach therefore
requires fewer modules to approximate the system dynamics.  %FIXME:
                                %JJB one of the known problems with
                                %what you are calling the "switching
                                %modular method" is blending as you
                                %move from one module / phase to the
                                %next.  I propose inserting therefore
                                %this sentence, but check if it is
                                %true in your system:
Because these modules also handle multiple phases of the project, we
also avoid the problem of blending and control-nonlinearities which
can happen when passing control between switched modules~\citep{Brand00,Miura09}.


Despite the many discussions of the modular approach, its application
in robotics is not yet wide-spread. One main challenge is how to
modularize a given task --- that is, how to decompose the task,
determining the number of modules their responsibilities. This is a
fundamental problem in the research of motion
primitives. \citet{kulic2008incremental} use a hierarchical clustering
method to extract primitives from human motion sequences. Different
cut-off parameters are tested to evaluate the trade-off effects
between facilitating quick group formation and introducing
misclassification. Our modularize approach is similar to this, but
goes one step further. We cluster the demonstration data with a
hierarchical method and regard each cluster as one module. Instead of
hand tuning the cut-off parameter, we determine its value by the
variance of the data. This provides us with a proper grouping of the
data, which can generate proper motor commands for control.

Figure.~\ref{fig:control} illustrates the workflow of our approach. To
the best of our knowledge, our work is the first realization of the
modular approach in learning an object manipulation task with a real
robot.


%We tackle this problem by a data driven approach: This solution can be applied to modularize many manipulation tasks. A similar clustering method has been applied to
% In our approach, the cut off parameter is determined by the variance of the data and hence avoid this step.

%Indeed, neuroscience evidence suggests that animal brain sum modular stimulates of muscles to generate different motions~\citep{mussa1994linear}.


%It is particularly useful to use learning to approximate internal models for motor control when the analytical model of the system dynamics is not easy to deduce.
%
%
%%Though the MOSAIC has gained a large concerns in neuroscience community, this approach is not widely used in robot control.
%
%Imitation learning has been an active topic of research in the last decade. It speeds up robot learning with the demonstrator showing the constraints of the tasks.

%The approach we take is a modular approach inspired from MOSAIC~\citep{haruno2001mosaic}. In MOSAIC, each dynamic system context is encoded by a pair of forward and inverse models. At each time step, the forward models estimate the current context and compare it with the actually context. The accuracy of the estimation decided the weight of the corresponding inverse models. The final motor command is the linear combination of the commands generated from each inverse models factored by their weights. The linear combination, rather than switch, of the models requires less number of models to approximate the system dynamics. Indeed, neuroscience evidence suggests that human brain mix modular stimulates of muscles to generate new motions. Our work focus on learning human's strategy of adaptive control on a modular base. To the best of our knowledge, this is the first realization of the multiple model control in an object manipulation task with a real robot, learnt from human demonstration.
%It can be used for solving nonlinear and non-stationary control problem.



%learning from human demonstration approach.


%
%Scientist has long been fascinated by human adaptive motor control ability and wondering the mechanism of it. One of the most evidently supported hypothesis is the multiple model~\citep{neilson1985acquisition}. %~\citep{haruno2001mosaic}



